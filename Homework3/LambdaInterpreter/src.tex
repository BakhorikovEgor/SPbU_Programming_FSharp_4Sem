\documentclass[a4paper,12pt]{article}
\usepackage[utf8]{inputenc}
\usepackage{amsmath}
\usepackage[russian]{babel}
\usepackage{amsfonts}

\begin{document}

\section*{Normalization}
\[
((\lambda a. (\lambda b. b) \; b) \; ((\lambda c. (c \; b)) \; (\lambda a. a))) \to_\beta
\]
\[
((\lambda b. b) \; ((\lambda c. (c \; b)) \; (\lambda a. a))) \to_\beta
\]
\[
((\lambda b. b) \; ((\lambda a. a) \; b)) \to_\beta
\]
\[
((\lambda b. b) \; b) \to_\beta b
\]

\noindent
Используем теорему Карри. Если дальше применять бета-редукцию форма терма не поменяется, выходит нормальной формы нет.

\section*{S K K}
\[
(\lambda x \; y \; z. x \; z \; (y \; z)) \; (\lambda x \; y. x) \; (\lambda x \; y. x) \to
\]
\[
(\lambda y \; z. (\lambda x \; y. x) \; z \; (y \; z)) \; (\lambda x \; y. x) \to
\]
\[
\lambda z. ((\lambda x \; y. x) \; z \; ((\lambda x \; y. x) \; z)) \to
\]
\[
\lambda z. ((\lambda x \; y. x) \; z \; (\lambda y. z)) \to
\]
\[
\lambda z. ((\lambda y. z) \; (\lambda y. z)) \to
\]
\[
\lambda z. z = I
\]

\end{document}
